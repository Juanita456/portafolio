\documentclass[12pt,a4paper]{article}
\usepackage[utf8]{inputenc}
\usepackage[spanish]{babel}
\usepackage{apacite}
\usepackage{hyperref}
\usepackage{amsmath}
\usepackage{amsfonts}
\usepackage{amssymb}
\usepackage{ragged2e}
\author{Dominguez Diaz Alma Juanita}
\title{Native and non-native applications.}

\begin{document}
\maketitle 
\section*{What are native applications?}
\justify
Native applications are those developed specifically for an operating system and require installation on the device.

Advantages:
\begin{itemize}
\item Performance: Native applications are faster and more efficient than other options because are more optimized to operate on a specific operating system.
\item Native applications have full access to device functionalities of the dispositive, like as camera, the microphone, or sensors.
\item Integration with the operating system: since native applications integrate seamlessly with the operating system can utilize all system's unique features and designs.
\end{itemize}

\justify Disadvantages:

\begin{itemize}
\item Development and maintenance can be expensive because they require specialized knowledge.

\item Native applications are not for other operating systems. 

\item It is necessary to download and install native applications on the device.
\end{itemize}

\section{What are not native applications?}
\justify Non-native applications consist of a single development that can be used across multiple operating systems and are also easily accessible through a web browser.

Advantages: 
\begin{itemize}
\item Non-native applications are faster and more cost-effective to develop, as they can share a codebase across multiple platforms (iOS, Android, web, etc.).

\item It is easier to maintain than non-native applications.

\item Web technologies such as JavaScript, HTML5, and CSS are used, which simplifies the implementation of certain functionalities.
\end{itemize}

Disadvantages:
\begin{itemize}
\item Non-native objects or applications are less efficient than native ones, as they lack full optimization for the operating system they run on. 

\item User experience can be incomplete.

\item By depending on web technologies, they may have restrictions to access certain device functionalities.

\item Non-applications may have more difficulty functioning without an Internet connection due to their dependence on web resources.

\item Compatibility issues with certain devices or operating systems versions.
\end{itemize}

\bibliographystyle{apacite}
\bibliography{bibliografia1}

\end{document}